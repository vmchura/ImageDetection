%--------------------------------------------------
% CPSC 202: Mathematical Tools for Computer Science
% Yale University
%
% Template for Homework Assignments
% by Yiding Hao
%--------------------------------------------------
\documentclass{cpsc202}

% Please put your information here.
\myname{Chura Quispe, Victor Miguel}
\myprofessor{Marc Masias}
\mysources{Your Sources}
\hwnumber{0}
\lstset{basicstyle=\ttfamily}

% \singlesided % Uncomment to print single-sided.
\graphicspath{{./media/}}
% Start typing your solution here.
\begin{document}

    \centerline{\Large\textbf{Deep Learning Project 2: Object detection}}

    Objective of the document: Details about the process of implementing algorithms for object detection.

    \large\textbf{Part I: Inference on test images and videos.}

    Overall the problem that I chose to solve is to detect butterflies.
    Firstly because in the search of datasets to images I've found this challenge: `Butterfly Detection' by Nuwe.io.
    But the challenge is actually classification, with the intention to practice even more I attempt the challenge with the knowledge learnt from the previous Project 1.
    \begin{center}
        \includegraphics[width=0.6\textwidth]{challenge_butterfly_classification}
        \includegraphics[width=0.3\textwidth]{challenge_result}
    \end{center}

    Back to this Project of Object Detection, I tried to use Yolo-11s on the images from the challenge I mentioned and a YouTube's video entirely of butterflies,
    the results are shown in the figure~\ref{fig:pre-trained}
    \begin{figure}
        \begin{subfigure}{.9\textwidth}
            \centering
            \includegraphics[width=.4\linewidth]{pretrained/negative_imagen_723}
            \includegraphics[width=.4\linewidth]{pretrained/negative_imagen_880}
            \caption{Negative class in the training data of the challenge}
            \label{fig:negative-pretrained}
        \end{subfigure}

        \begin{subfigure}{.9\textwidth}
            \centering
            \includegraphics[width=.4\linewidth]{pretrained/positive_imagen_45}
            \includegraphics[width=.4\linewidth]{pretrained/positive_imagen_888}
            \caption{Positive class in the training data of the challenge}
            \label{fig:positive-pretrainied}
        \end{subfigure}

        \begin{subfigure}{.9\textwidth}
            \centering
            \includegraphics[width=.4\linewidth]{pretrained/photogram_16}
            \includegraphics[width=.4\linewidth]{pretrained/photogram_93}
            \caption{Images of the YouTube video}
            \label{fig:video-pretrainied}
        \end{subfigure}
        \caption{Images results using only the pre-trained Yolo 11s}
        \label{fig:pre-trained}
    \end{figure}

    \newpage
    \large\textbf{Part II: Transfer learning to detect one novel class}
    Following the same problem, to detect butterflies, firstly I've chosen the dataset with only 2096 images.
    \begin{center}
        \includegraphics[width=0.8\textwidth]{trained_small_butterflies/dataset_simple}
    \end{center}
    The dataset only has a pre-process of resizing to 640.
    The relevant metrics for training and validation are shown in the figure~\ref{fig:results-small}.
    From the charts, specially from box\_loss and cls\_loss I can see that the model learns very rapidly from the train data and it also the learning is successful against the validation set.
    Since we are measuring only one class, precision and recall also are almost 1.0.
    The model looks perfect, but if I test it against the dataset from the Challenge, I've only got a poor accuracy of approximately 0.5.
    Using the inference of the model to infer from the images of the challenge and YouTube's video are in the figure~\ref{fig:trained_small_butterflies}
    It looks like the images from training and validation are quite similar, that is why it looks like there is an overfitting.
    \begin{figure}
        \begin{center}
            \includegraphics[width=0.8\textwidth]{trained_small_butterflies/results}
            \includegraphics[width=0.4\textwidth]{trained_small_butterflies/confusion_matrix_simple}
        \end{center}
        \caption{Result of training of the small dataset of butterflies.}
        \label{fig:results-small}
    \end{figure}
    \begin{figure}
        \begin{subfigure}{.9\textwidth}
            \centering
            \includegraphics[width=.4\linewidth]{trained_small_butterflies/negative_imagen_723}
            \includegraphics[width=.4\linewidth]{trained_small_butterflies/negative_imagen_880}
            \caption{Negative class in the training data of the challenge}
            \label{fig:negative-trained_small_butterflies}
        \end{subfigure}

        \begin{subfigure}{.9\textwidth}
            \centering
            \includegraphics[width=.4\linewidth]{trained_small_butterflies/positive_imagen_45}
            \includegraphics[width=.4\linewidth]{trained_small_butterflies/positive_imagen_888}
            \caption{Positive class in the training data of the challenge}
            \label{fig:positive-trained_small_butterflies}
        \end{subfigure}

        \begin{subfigure}{.9\textwidth}
            \centering
            \includegraphics[width=.4\linewidth]{trained_small_butterflies/photogram_16}
            \includegraphics[width=.4\linewidth]{trained_small_butterflies/photogram_93}
            \caption{Images of the YouTube video}
            \label{fig:video-trained_small_butterflies}
        \end{subfigure}
        \caption{Images results using transfer learning using the small dataset of butterflies}
        \label{fig:trained_small_butterflies}
    \end{figure}

\end{document}